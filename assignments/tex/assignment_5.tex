% Copyright Patrick Hall 2021

\documentclass[fleqn]{article}
\renewcommand\refname{}
\title{Responsible Machine Learning\\\Large{Assignment 5}\\\Large{10 points}}
\author{\copyright Patrick Hall 2021}

\usepackage{graphicx}
\usepackage{fullpage}
\usepackage{pdfpages}
\usepackage{amsmath}
\usepackage{amssymb}
\usepackage{mathtools}
\usepackage{MnSymbol}
\usepackage{enumerate}
\usepackage{setspace}
\usepackage[colorlinks, breaklinks=true]{hyperref} 
\usepackage{float}
\usepackage{caption}
\usepackage{subcaption}
\usepackage{multicol}
\usepackage{color}
\usepackage{listings}
\usepackage{csvsimple}
\usepackage{algorithm}
\usepackage{algorithmic}
\usepackage{verbatim}
\usepackage{mdframed}
\usepackage{changepage}
\usepackage[top=1in, bottom=1in, left=1in, right=1in]{geometry}

\begin{document}

\maketitle

\noindent In Assignment 5, you will work with your group to debug your best model following the instructions below. A \href{https://nbviewer.jupyter.org/github/jphall663/GWU_rml/blob/master/assignments/assignment_5/assign_5_template.ipynb?flush_cache=true}{template} has been provided with examples of simple sensitivity and residual analysis exercises. For those of you who use Python virtual environments, a basic \href{https://github.com/jphall663/GWU_rml/blob/master/assignments/requirements.txt}{\texttt{requirements.txt}} file is also available for the template.\\

\noindent Please let me know immediately if you find typos or mistakes in this assignment or related materials. 

\section{Test How this Lending Model Performs in Recession Conditions.}

Cells 7--8 demonstrate a basic stress-testing exercise, in which recession conditions are simulated and model performance is re-estimated under these conditions. Generally, lending model performance degrades quickly when recession conditions arise. You should see that our EBM models are no exception to this rule.\\

\section{Conduct Residual Analysis and Remediate Discovered Bugs.}

Cells 9--15 use a basic residual analysis procedure to find outliers and identify a fundamental problem with our data and EBM model. Once these problems are identified, you should be able to increase your model performance by accounting for them.\\

\section{Submit Code Results.}

Your deliverable for this assignment is to update your group's GitHub repository to reflect this debugging exercise. Stress-testing is worth 5 points. Remediating your model by removing outliers and handling data imbalance, to increase validation AUC, is worth 5 points.\\

\noindent \textbf{Your deliverables are due Saturday, June 26\textsuperscript{th}, at 11:00:00 AM ET.}\\

\noindent Note that you may also improve Assignment 1 or Assignment 3 scores throughout the Summer I Session to improve your ranking, your Assignment 1 grade, your Assignment 3 grade, and your final project grade.

\end{document}

